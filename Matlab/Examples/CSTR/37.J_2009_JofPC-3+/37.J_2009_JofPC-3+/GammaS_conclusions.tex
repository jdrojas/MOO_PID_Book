%-----------------------------------------
%
\section{Conclusions}
\label{conclusions}
%
%-------------------------------------------

In process control it is very usual to have changes in the
set-point, as well as in the disturbance. This causes the need to
face with both servo and regulatory control problems. For 1-DoF
PID controllers, when the tuning objective is different to the
real system operation, a degradation in the performance is
expected and it can be evaluated. A reduction in the overall
Performance Degradation can be obtained by searching an
\emph{intermediate} controller between the optimal ones proposed
for set-point and load-disturbance tunings.

Autotuning formulae have been presented with the aim to minimize
the Weighted Performance Degradation, expressed as a combination
depending of the balance between the total time that the system
operates in servo and regulation modes. This is the main
contribution of this paper because it is a novel feature that
allows to select the tuning according to a general qualitative
specification of the control system operation.

Results are given for PID controllers, in order to get results
closer to industrial applications. The examples have shown the
improvement obtained with each one of the
$\overline{\gamma}_{\alpha}-autotuning$ cases.

Even if the results were presented and exemplified using the ISE
performance criteria, it could be possible to reproduce a similar
methodology to other PID controllers, like the one that uses
derivative action is applied just to process output, or to other
PID tunings with different performance objectives.
