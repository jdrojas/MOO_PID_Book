%-----------------------------------------
%
\section{Introduction}
%
%-------------------------------------------

Proportional-Integrative-Derivative (PID) {contro\-llers} are with
no doubt the most extensive option that can be found on industrial
control applications \cite{astromCEP2001}. Their success is mainly
due to its simple structure and to the physical meaning of the
corresponding three parameters (therefore making manual tuning
possible). This fact makes PID control easier to understand by the
control engineers than other most advanced control techniques. In
addition, the PID controller provides satisfactory performance in
a wide range of practical situations.

During the last years, in fact since the initial work of Ziegler
and Nichols \cite{zieglernichols42}, much work has been done
developing methods to determine the PID controller parameters (see
for example
\cite{astromJPC2004,skogestadJPC2003,kristianssonJPC2006}).
O'Dwyer \cite{odwyer2003} presents a collection of tuning rules
for PID controllers, which show their abundance.

Within the wide range of approaches to autotuning, optimal methods
have received special interest. These methods provide, given a
simple model process description -such as a
First-Order-Plus-Dead-Time (FOPDT) model- settings for optimal
closed-loop responses \cite{zhuangAthertonIEE1993}.

For One-Degree-of-Freedom (1-DoF) controllers, it is usual to
relate the tuning method to the expected operation mode for the
control system, known as \emph{servo} or \emph{regulation}.
Therefore, controller settings can be found for optimal set-point
or load-disturbance responses. This fact allows better performance
of the controller when the control system operates on the selected
tuned mode but, a degradation in the performance is expected when
the tuning and operation modes are different. Obviously there is
always the need to choose one of the two possible ways to tune the
controller, for set-point tracking or load-disturbances rejection.
In the case of 1-DoF PID, tuning can be optimal just for one of
the two operation modes. The main problem, about the Performance
Degradation analysis for both tuning modes, was previously
formulated in \cite{arrietaCSC2007} and some approaches related to
tuning methods and autotuning have been proposed in
\cite{arrietaMED2007,arrietaCDC2007}.

What is provided in this paper is a continuation of the these
ideas in order to find an \emph{intermediate} tuning for the
controller that improves the overall performance of the system,
considered as a \emph{trade-off} between servo and regulation
operation modes. The settings are determined from the combination
of the optimal ones for set-point and load-disturbance, presented
in \cite{zhuangAthertonIEE1993}, and taking into account the
balance between the importance of each one of the operation modes
for the control system (servo or regulation). The optimization is
here performed using genetic algorithms \citep{mitchell1998}.

The proposed new method considers a 1-DoF PID controller as an
alternative when an \emph{explicit} 2-DoF PID controller is not
available. It should be remembered that for the
Two-Degree-of-Freedom (2-DoF) PID controller, tuning is usually
optimal for regulation operation and suboptimal for servo-control,
where this suboptimal behavior is achieved using a set-point
weighting factor as an extra tuning parameter that gives the
second Degree-of-Freedom, to improve the tracking action. Also,
sometimes is not strictly necessary, or not justified, to increase
the number of the tuning parameters in contrast to the benefits
that could be obtained. It could be stated that the proposed
\emph{intermediate} tuning is a particular case that results in a
suboptimal tuning, when both operation modes may happen.

The paper is organized as follows. Next section introduces the
general problem formulation, with some related concepts. Section
\ref{methodology} presents the \emph{intermediate} tuning between
the parameters of both operation modes in such a way that a
Weighted Performance Degradation ($WPD$) is minimized; the results
are generalized in terms of an autotuning procedure that is
presented in Section \ref{autotuning}. Some examples are shown in
Section \ref{example} and the conclusions are drawn in Section
\ref{conclusions}.
