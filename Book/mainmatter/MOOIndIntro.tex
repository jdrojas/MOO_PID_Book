\chapter{Introduction}
%\begin{refsection}
\label{sec:Antecedentes}
The design of control systems always has to consider multiple and possibly conflicting design objectives. Under this perspective, the task of the engineer in charge of the control system, becomes to find the optimal point of compromise within this set of distinct objectives \cite{Garpinger2012}.

The most used control algorithm in industry is the \gls{pid}. This type of algorithm is used in a wide variety of applications, due to its limited number of parameters, its ease of implementation and its robustness \cite{Astrom2005} and represents an area of active study since the first tuning methodology was proposed in the 1940s \cite{Ziegler1942}. In the case of this particular project, in order to have a direct impact on the industry and the research community in process control systems, it will focus on the problem of the tuning of the parameters of controllers \gls{pid}.

It is common that the problem of tuning the parameters of industrial controllers is posed as an optimization problem. When all the objectives need to be taken into account at the same time, this problem becomes a multivariate multiobjective optimization problem. In the particular case of industrial \gls{pid} controllers, this problem is also non-lineal and (possibly) non convex, therefore, the problem at hand is not trivial.

Regardless of the methodology to be used, it is generally computationally expensive to solve a multiobjective optimization problem, which can lead to a scenario of multiple solutions equally optimal, so that in addition to solving the optimization problem, the control engineer, ends up with the extra responsibility of entering into a decision phase a posteriori to finally choose the best set of parameters for its specific application.

In this sense, \gls{moo} tuning of \gls{pid} controllers remains as an open research subject, even though it has bee studied for several decades. For example, in \cite{Seaman1994} a type of \gls{moo} is used to tune \gls{pid} controllers in a plastic injection molding process. In \cite{Abbas1995}, an algorithm based on several optimizations is proposed to find the optimal parameters of a \gls{pid} controller; this algorithm took into account several variables such as stationary error, rise time, overrun, settling time and maximum controller output within the feedback loop. More recently, bio-inspired techniques such as neural networks, fuzzy logic or genetic algorithms have been used to solve the optimization problem \cite{Reynoso-Meza2012b}. In \cite{Bagis2011}, A Tabu search algorithm is used to tune \gls{pid} controllers in real time, based in a set of closed loop specifications and a cost function. In \cite{Chiha2012} the \gls{moop} for \gls{pid} controllers is solved using the ant colony approach, this methodology tries to simulate the behavior of real ants when they are looking of the shortest path to a given objective.

Beside bio-inspired methods for \gls{moop}, there are several methdologies that transform the \gls{moop} into a single function optimization problem, by rewriting the problem with extra constrains. The simplest method is the \gls{ws} \cite{Marler2004}. With the \gls{ws} method, the multiobjective cost function is transform in a one dimensional function using a weighted sum that give a greater relative weight to a function in comparison to the others. For each set of weight values a different optimal solution is found for the same problem. The set of all solutions is part of the Pareto front \cite{Marler2004}. The Pareto front corresponds to all equally optimal solutions for a \gls{moop}. The problem with the \gls{ws} method is that, although the results obtained are from the Pareto front, it is not possible to satisfactorily construct the entire front \cite{Das1997,Messac2000,Marler2010}.

In order to obtain the Pareto front correctly, other methodologies have emerged that surpass the \gls{ws}. The \gls{nbi} method \cite{Das1998} consist in rewriting the optimization problem so that the feasible area is shortened by an equality constraint that depends on an extra parameter. The solution of this new problem will terminate at the Pareto border and by varying this extra parameter, it is possible to find the Pareto front so that each found point is equally spaced at the front. This feature is very useful since it gives an overall idea of the shape of the front. \gls{nbi} has been applied to the tuning of controllers in \cite{Gambier2009} where the controller is selected taking into account different performance indexes like the integral of the squared error (ISE), the integral time-weighted squared error (ITSE) and the integral of the squared time-weighted squared error (ISTSE). Other methodology similar to \gls{nbi} is the \gls{nnc} \cite{Messac2003}, which converts the \gls{moop} in a single function optimization with an extra inequality constraint.

It should be noted that these methodologies have also been used in other areas apart from the control of industrial processes. A few examples of the areas in which it has been applied are: calculation of optimal power flow in power systems \cite{Roman2006}  and distributed generation planning \cite{Zangeneh2007},  for the control of biochemical processes \cite{Logist2009}, circuit analysis \cite{Stehr2003}, development of optimal supply strategies for the participants of oligopolistic energy markets \cite{Vahidinasab2010}.

Although in the area of process control, there are examples of the use of these algorithms \cite{Gambier2009}, they do not take into account the different sources of disturbance to the system, but are different measures of performance to the same source of disturbance (the change in reference), and using a simple \gls{pid} controller of a single degree of freedom.
%\printbibliography
%\end{refsection}