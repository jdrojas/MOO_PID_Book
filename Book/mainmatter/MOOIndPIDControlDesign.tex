\chapter{PID Controller Design}

\section{PID Controller Tuning}
The appropriate selection of the tuning parameters is one of the most important steps in the definition of a PID based control loop. This is known as the PID tuning. The selection of the PID controller parameters should be made according to the available knowledge of the process dynamics and stated performance specifications in terms of tracking and disturbance attenuation as well as desired robustness. One of the aspects that makes PID control specially appealing is the clear physical meaning associated to each one of its parameters. \\

Numerous studies have been made to develop assignment rules to specify PID parameters on the basis of characteristics of the process being controlled. The collected information about the process to be controlled can, in one form or another be assimilated to a model of the process. This can be referred either as a parametric process model (or, in other words, a form suitable for analyzing and simulating the closed-loop system), or concrete process data and/or measurements that in a suitable way can be directly employed to determine the PID controller parameters. There are many representative sources that can be consulted for details on a wide variety of alternative tuning rules  \cite{odwyer2006}, \cite{vilanova2012}. \\

In what follows, a short presentation of the most usual existing methods for PID tuning are presented. The presentation will not be deep into details because it is not the purpose of this book to explore existing tuning approaches, material that, on the other hand, can be accessed on numerous sources. However, as our approach deals with a conceptual approach that joins different specifications, it is 

\subsection{Analytical Tuning Methods}
To do
\subsection{Tuning based on Minimisation of a Performance Criterions}
To do
\subsection{Tuning Rules for Robustness}
To do
%
\section{Formalization of PID tuning as a multiobjective optimization problem}
\label{sec:FormPIDMOOP}
To do
\subsection{Cost functions and constraints selection}
\label{sec:CostFunSelec}
To do
\subsection{PID tuning problem formulation for integral cost functions}
\label{sec:CostProbPID}
To do