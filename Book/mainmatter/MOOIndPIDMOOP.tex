\chapter{PID tuning as a multi-objective optimization method}
\label{chap:PIDMOOP}

%-----------------------------
\section{Solution of the multi-objective optimization tuning}
\label{sec:SolMOOP}
When solving the \gls{moop} presented in \eqref{eq:probmoo} for different normalized plants, one is able to find a family of Pareto fronts. The script used to find these fronts can be found in the companion software for this book\footnote{\textbf{PONER EL NOMBRE DEL SCRIPT}}.

Defining the normalized variable $\hat{s}=T s$, the time delay for the normalized plant, $\tau_0$, becomes:
\begin{equation}
\tau_0 = \frac{L}{T},
\label{eq:tauNorm}
\end{equation}
%

Then, one is able to find the corresponding Pareto front for the normalized plant, that represents many combinations of lag time and dead time.
%

The Pareto front was found for each normalized plant with approximately 1000 points for each front. Notices that each of these points represents a different tuning (and since the optimization was done for \gls{2dof} controllers, each point has a different value for $\kappa_p$, $\tau_i$, $\tau_d$ and $\beta$), the total possible Pareto optimal \gls{pid} controllers found, reaches approximately 100 000 different controllers for all the totality of the plants. All these controller tunings, can also be found in the companion software of this book as comma separated values.

\section{Viability for tuning rules}
\label{sec:TuningRulesMOOP}
In this section, an example of how to use the gathered data from the Pareto front is presented in order to find a tuning rule. The results that are presented here where first documented in \cite{Moya2017}.

For this particular example, the variable $a$ takes values from $0$ to $1$ in $0.1$ steps and $\tau_0$ takes values from $1$ to $2$ in $0.1$ steps and $M_{s,max}=2$. However, the complete set of data also has values for $\tau_0 < 1$.

The methodology after finding the data is to perform a curve fitting procedure to find equations useful to compute the value of $\kappa_p$, $\tau_i$, $\tau_d$ and $beta$ as a function of $a$ and $\tau_0$ and a factor of degradation of the cost functions.

Notice that the idea is that, knowing the model of the plant, the values of the controller parameters can be computed without needing to perform all the optimization, and also the decision maker can also select the weight for each cost function in order to find a single set of parameters.

This idea of "allowed degradation" is now introduced. Considered that $J_{di}$ and $J_{do}$ are normalized as:
%
\begin{align}
\delta &= \frac{J_{di}(\theta)-J_{di, min}(\theta)}{J_{di,max}(\theta)-J_{di,min}(\theta)},\label{eq:delta}\\
\gamma &= \frac{J_{do}(\theta)-J_{do, min}(\theta)}{J_{do,max}(\theta)-J_{do,min}(\theta)},\label{eq:gamma}
\end{align}
%
where both $0 \le \delta \le 1$ and $0 \le \gamma \le 1$. Then, these variables can be understood as the degradation of the function, considering the minimum value of the cost function as its optimal. Then a value of $\delta=1$ represents a degradation of 100\% of the $J_{di}$ cost function. It is important to notice that the Pareto front is constructed from three different cost functions. Therefore, if one select the value of the allowed degradation for two functions (in this case, $\delta$ and $\gamma$), the logical step is to choose the lowest value of $J_r$ that complies with the maximum degradation of the other two functions. Then for example,  if $\delta=\gamma=1$, which means that the decision maker is willing to allowed a complete degradation of $J_{di}$ and $J_{do}$, the resulting tuning is expected to represent the optimal tuning for servo control.

Now, it is important to understand that the ``degraded'' tuning, is also optimal in the Pareto sense, because all found tunings are optimal. Therefore, in these frame, a degraded tuning does not mean a ``bad'' tuning, it is just the result of a choice decision when selecting the final optimal controller. In all Pareto decisions, a compromise has to be made when selecting the final solution.

The work done to find the tuning rules, summed up to almost two hundred and twenty regressions for all values of $a$ and $\tau_0$ in order to find the complete set of parameters $\bm{\theta}$.

After different heuristic tests, the regression analysis showed that a second order fit gave the best results for $\kappa_p$, $\tau_i$ and $\tau_d$,  while  a first order fit for $\beta$ was enough to model the variation of this parameter. 

The tuning rule for all controller parameters are proposed to be as:  
%
\begin{align}
\kappa_p &= p_{00}+p_{01}\cdot\gamma+p_{02}\cdot\delta\nonumber\\
&\quad + p_{03}\cdot\gamma^2+p_{04}\cdot\gamma\cdot \delta+p_{05}\cdot\delta^2,\label{E:eqkp}\\
%
\tau_i &= p_{10}+p_{11}\cdot\gamma+p_{12}\cdot\delta\nonumber\\
&\quad + p_{13}\cdot\gamma^2+p_{14}\cdot\gamma\cdot \delta+p_{15}\cdot\delta^2,\label{E:eqTi}\\
%
\tau_d &= p_{20}+p_{21}\cdot\gamma+p_{22}\cdot\delta\nonumber\\
&\quad+p_{23}\cdot\gamma^2+p_{24}\cdot\gamma\cdot \delta+p_{25}\cdot\delta^2,\label{E:eqTd}\\
%
\beta &=p_{30}+p_{31}\cdot\gamma+p_{32}\cdot\delta,\label{E:eqbeta}
\end{align}
%

The coefficients $p_{ij}$, where $i=\{0,1,2,3\}$ and $j=\{0,1,2,3,4,5\}$, depend on $a$ and $\tau_0$. The corresponding fits of $\kappa_p$, $\tau_i$, $\tau_d$ and $\beta$, are shown in Fig.~\ref{F:cftoolkp}. \ref{F:cftoolTi}, \ref{F:cftoolTd} and \ref{F:cftoolbeta} for $a=0.1$ and $\tau_{0}=1$.   % a second order surface was found to be the best fit for all eleven fits overall, as shown in figure \ref{F:firstfit}.
%
\begin{figure}[tb]
	\centering
	\includegraphics[width=\columnwidth]{kpfit2.png}
	\caption{Second order fit for $\kappa_p$ when $a=0.1$ and $\tau_{0}=1$}
	\label{F:cftoolkp}
\end{figure}
%
\begin{figure}[tb]
	\centering
	\includegraphics[width=\columnwidth]{Tifit2.png}
	\caption{Second order fit for $\tau_i$ when $a=0.1$ and $\tau_0=1$}
	\label{F:cftoolTi}
\end{figure}
%
\begin{figure}[tb]
	\centering
	\includegraphics[width=0.5\textwidth]{Tdfit2.png}
	\caption{Second order fit for $\tau_d$ when $a=0.1$ and $\tau_0=1$}
	\label{F:cftoolTd}
\end{figure}
%
\begin{figure}[tb]
	\centering
	\includegraphics[width=0.5\textwidth]{betafit2.png}
	\caption{First order fit for $\beta$ when $a=0.1$ and $\tau_0=1$}
	\label{F:cftoolbeta}
\end{figure}

It is important to notice an important caveat. In those figures, the values that belong to the computed Pareto are shown as dots, while the corresponding regression is ploted as a 3D surface. It can be noticed that the domain of the regressions is larger than the actual results of the Pareto. Even thought the fitting is good (around $R=0.9$), the regression represents interpolation and extrapolation from the real data. Therefore, it is important too check how well the regression works and to not exceed the limits where it yields good results.

Going back to the $p_{ij}$, it has to be notice that the value of these parameters, depends on the model of the plant. Therefore, it is required to find another set of regressions over these parameters in terms of $a$ and $\tau_0$. Therefore, a curve fitting procedure is also required for each $p_{ij}$. As an example of these regressions, 
%
\begin{figure}[tb]
	\centering
	\includegraphics[width=0.5\textwidth]{./a00fit2.png}
	\caption{Second order fit for $p_{00}$ in $\kappa_p$}
	\label{F:coeff}
\end{figure}
%
Fig.~\ref{F:coeff} shows the result for $p_{00}$ parameter as a function of $a$ and $\tau_0$. The selected fit for every coefficient in the range of $1\leq \tau_0 \leq2$, was also a second order polinomial. The equation that is considered has the form:
%
\begin{equation}
p_{ij} = b_{j0}+b_{j1}a+b_{j2} \tau_0+b_{j3}a^2+b_{j4}a \tau_0+b_{j5}\tau_0^2 .
\label{E:coeff}
\end{equation}

Another two hundred and twenty regressions were made for all $p_{ij}$. The results for every coefficient are shown in Table~\ref{T:T1} for $kappa_p$, Table~\ref{T:ti} for the integral time, Table~\ref{T:td} for the derivative time and Table~\ref{T:beta} for $\beta$.

% Please add the following required packages to your document preamble:
% \usepackage{multirow}
%\begin{table}
%\centering
%\caption{Parameters for $\kappa_p$.}
%\label{T:T1}
%\begin{tabular}{|c|l|l|}
%\hline
%\multicolumn{3}{|c|}{$\kappa_p$ coefficients}           \\ \hline
%$p_{ij}$                  & \multicolumn{2}{c|}{$b_{ik}$} \\ \hline
%\multirow{6}{*}{$p_{00}$} & $b_{00}$      & 1.8203        \\ \cline{2-3} 
%                     & $b_{01}$      & 0.12765       \\ \cline{2-3} 
%                     & $b_{02}$      & -1.0484       \\ \cline{2-3} 
%                     & $b_{03}$      & 0.27085       \\ \cline{2-3} 
%                     & $b_{04}$      & -0.15141      \\ \cline{2-3} 
%                     & $b_{05}$      & 0.25505       \\ \hline
%\multirow{6}{*}{$p_{01}$} & $b_{10}$      & 0.32832       \\ \cline{2-3} 
%                     & $b_{11}$      & 0.22439       \\ \cline{2-3} 
%                     & $b_{12}$      & -0.26761      \\ \cline{2-3} 
%                     & $b_{13}$      & -0.022374     \\ \cline{2-3} 
%                     & $b_{14}$      & -0.068708     \\ \cline{2-3} 
%                     & $b_{15}$      & 0.076273      \\ \hline
%\multirow{6}{*}{$p_{02}$} & $b_{20}$      & 0.29122       \\ \cline{2-3} 
%                     & $b_{21}$      & -0.12878      \\ \cline{2-3} 
%                     & $b_{22}$      & -0.25003      \\ \cline{2-3} 
%                     & $b_{23}$      & 0.10461       \\ \cline{2-3} 
%                     & $b_{24}$      & 0.0048329     \\ \cline{2-3} 
%                     & $b_{25}$      & 0.059478      \\ \hline
%\multirow{6}{*}{$p_{03}$} & $b_{30}$      & 0.042733      \\ \cline{2-3} 
%                     & $b_{31}$      & -0.52048      \\ \cline{2-3} 
%                     & $b_{32}$      & -0.25437      \\ \cline{2-3} 
%                     & $b_{33}$      & 0.47345       \\ \cline{2-3} 
%                     & $b_{34}$      & -0.11069      \\ \cline{2-3} 
%                     & $b_{35}$      & 0.078691      \\ \hline
%\multirow{6}{*}{$p_{04}$} & $b_{40}$      & -0.077455     \\ \cline{2-3} 
%                     & $b_{41}$      & 0.61083       \\ \cline{2-3} 
%                     & $b_{42}$      & 0.24951       \\ \cline{2-3} 
%                     & $b_{43}$      & -0.60316      \\ \cline{2-3} 
%                     & $b_{44}$      & 0.19685       \\ \cline{2-3} 
%                     & $b_{45}$      & -0.071314     \\ \hline
%\multirow{6}{*}{$p_{05}$} & $b_{50}$      & -0.41226      \\ \cline{2-3} 
%                     & $b_{51}$      & -0.24733      \\ \cline{2-3} 
%                     & $b_{52}$      & 0.29554       \\ \cline{2-3} 
%                     & $b_{53}$      & 0.080236      \\ \cline{2-3} 
%                     & $b_{54}$      & 0.013411      \\ \cline{2-3} 
%                     & $b_{55}$      & -0.091089     \\ \hline
%\end{tabular}
%\end{table}
%
\begin{table}[tb]
	\centering
	\caption{Coefficients for $\kappa_p$.}
	\label{T:T1}
	\begin{tabular}{@{}cll|cll@{}}
		\hline
		%\multicolumn{6}{c}{$\kappa_p$ coefficients}           \\ \hline
		$p_{ij}$                  & \multicolumn{2}{c}{$b_{ik}$} & $p_{ij}$ & \multicolumn{2}{c}{$b_{ik}$}\\
		\hline
		\multirow{6}{*}{$p_{00}$} & $b_{00}$  & $1.820$ &   \multirow{6}{*}{$p_{01}$} & $b_{10}$ & $0.328$ \\ % \cline{2-3} \cline{5-6}
		& $b_{01}$      & $0.128$   &  	& $b_{11}$      & $0.224$ \\ % \cline{2-3} \cline{5-6}
		& $b_{02}$      & $-1.048$   &	& $b_{12}$      & $-0.268$ \\ % \cline{2-3} \cline{5-6}
		& $b_{03}$      & $0.270$   &	& $b_{13}$      & $-0.022$ \\ % \cline{2-3} \cline{5-6}
		& $b_{04}$      & $-0.151$  & 	& $b_{14}$      & $-0.069$  \\ % \cline{2-3} \cline{5-6}
		& $b_{05}$      & $0.255$   &	& $b_{15}$      & $0.076$    \\ \hline
		%
		\multirow{6}{*}{$p_{02}$} & $b_{20}$  & $0.291$	& \multirow{6}{*}{$p_{03}$} & $b_{30}$ & $0.043$ \\ % \cline{2-3} \cline{5-6}
		& $b_{21}$      & $-0.129$   & & $b_{31}$      & $-0.520$  \\ % \cline{2-3} \cline{5-6}
		& $b_{22}$      & $-0.250$   & & $b_{32}$      & $-0.254$  \\ % \cline{2-3} \cline{5-6}
		& $b_{23}$      & $0.105$    & & $b_{33}$      & $0.473$   \\ % \cline{2-3} \cline{5-6}
		& $b_{24}$      & $0.005$  & & $b_{34}$      & $-0.111$  \\ % \cline{2-3} \cline{5-6}
		& $b_{25}$      & $0.059$   & & $b_{35}$      & $0.079$  \\ \hline
		%
		\multirow{6}{*}{$p_{04}$} & $b_{40}$      & $-0.077$  & \multirow{6}{*}{$p_{05}$} & $b_{50}$      & $-0.412$   \\ %\cline{2-3} \cline{5-6}
		& $b_{41}$      & $0.611$     &  & $b_{51}$      & $-0.247$\\ %\cline{2-3} \cline{5-6}
		& $b_{42}$      & $0.249$     &  & $b_{52}$      & $0.296$\\ % \cline{2-3} \cline{5-6}
		& $b_{43}$      & $-0.603$    &  & $b_{53}$      & $0.080$\\ %\cline{2-3} \cline{5-6}
		& $b_{44}$      & $0.197$     &  & $b_{54}$      & $0.013$\\ % \cline{2-3} \cline{5-6}
		& $b_{45}$      & $-0.071$   &  & $b_{55}$      & $-0.091$\\
		\hline
	\end{tabular}
\end{table}
%
\begin{table}[tb]
	\centering
	\caption{Coefficients for $\tau_i$.}
	\label{T:ti}
	\begin{tabular}{@{}cll|cll@{}}
		\hline
		%\multicolumn{6}{c}{$\kappa_p$ coefficients}           \\ \hline
		$p_{ij}$                  & \multicolumn{2}{c}{$b_{ik}$} & $p_{ij}$ & \multicolumn{2}{c}{$b_{ik}$}\\
		\hline
		\multirow{6}{*}{$p_{10}$} & $b_{00}$  & $0.591$
		&   \multirow{6}{*}{$p_{11}$} & $b_{10}$ & $-0.408$
		\\ % \cline{2-3} \cline{5-6}
		& $b_{01}$      & $0.559$   &  	& $b_{11}$      & $0.640$ \\ % \cline{2-3} \cline{5-6}
		& $b_{02}$      & $0.545$   &	& $b_{12}$      & $0.855$ \\ % \cline{2-3} \cline{5-6}
		& $b_{03}$      & $0.017$   &	& $b_{13}$      & $-0.238$ \\ % \cline{2-3} \cline{5-6}
		& $b_{04}$      & $0.045$  & 	& $b_{14}$      & $-0.0024$  \\ % \cline{2-3} \cline{5-6}
		& $b_{05}$      & $-0.028$   &	& $b_{15}$      & $-0.193$    \\ \hline
		%
		\multirow{6}{*}{$p_{12}$} & $b_{20}$  & $1.718$
		& \multirow{6}{*}{$p_{13}$} & $b_{30}$ & $1.297$ \\ % \cline{2-3} \cline{5-6}
		& $b_{21}$      & $0.652$   & & $b_{31}$      & $-0.423$  \\ % \cline{2-3} \cline{5-6}
		& $b_{22}$      & $-1.160$   & & $b_{32}$      & $-2.095$  \\ % \cline{2-3} \cline{5-6}
		& $b_{23}$      & $-0.855$    & & $b_{33}$      & $1.226$   \\ % \cline{2-3} \cline{5-6}
		& $b_{24}$      & $-0.719$  & & $b_{34}$      & $-1.041$  \\ % \cline{2-3} \cline{5-6}
		& $b_{25}$      & $0.363$   & & $b_{35}$      & $0.649$  \\ \hline
		%
		\multirow{6}{*}{$p_{14}$} & $b_{40}$      & $-0.077$  & \multirow{6}{*}{$p_{15}$} & $b_{50}$      & $-1.346$   \\ %\cline{2-3} \cline{5-6}
		& $b_{41}$      & $0.621$     &  & $b_{51}$      & $-1.148$\\ %\cline{2-3} \cline{5-6}
		& $b_{42}$      & $0.277$     &  & $b_{52}$      & $1.224$\\ % \cline{2-3} \cline{5-6}
		& $b_{43}$      & $-1.193$    &  & $b_{53}$      & $-0.218$\\ %\cline{2-3} \cline{5-6}
		& $b_{44}$      & $1.030$     &  & $b_{54}$      & $0.512$\\ % \cline{2-3} \cline{5-6}
		& $b_{45}$      & $-0.025$   &  & $b_{55}$      & $-0.572$\\
		\hline
	\end{tabular}
\end{table}
%
\begin{table}[tb]
	\centering
	\caption{Coefficients for $\tau_d$.}
	\label{T:td}
	\begin{tabular}{@{}cll|cll@{}}
		\hline
		%\multicolumn{6}{c}{$\kappa_p$ coefficients}           \\ \hline
		$p_{ij}$                  & \multicolumn{2}{c}{$b_{ik}$} & $p_{ij}$ & \multicolumn{2}{c}{$b_{ik}$}\\
		\hline
		\multirow{6}{*}{$p_{20}$} & $b_{00}$  & $0.111$
		&   \multirow{6}{*}{$p_{21}$} & $b_{10}$ & $-0.0076$
		\\ % \cline{2-3} \cline{5-6}
		& $b_{01}$      & $0.450$   &  	& $b_{11}$      & $-0.163$ \\ % \cline{2-3} \cline{5-6}
		& $b_{02}$      & $0.274$   &	& $b_{12}$      & $-0.212$ \\ % \cline{2-3} \cline{5-6}
		& $b_{03}$      & $-0.025$   &	& $b_{13}$      & $0.154$ \\ % \cline{2-3} \cline{5-6}
		& $b_{04}$      & $-0.069$  & 	& $b_{14}$      & $-0.074$  \\ % \cline{2-3} \cline{5-6}
		& $b_{05}$      & $0.003$   &	& $b_{15}$      & $0.0026$    \\ \hline
		%
		\multirow{6}{*}{$p_{22}$} & $b_{20}$  & $-0.238$	& \multirow{6}{*}{$p_{23}$} & $b_{30}$ & $-0.237$ \\ % \cline{2-3} \cline{5-6}
		& $b_{21}$      & $0.105$   & & $b_{31}$      & $-0.938$  \\ % \cline{2-3} \cline{5-6}
		& $b_{22}$      & $-0.016$   & & $b_{32}$      & $1.121$  \\ % \cline{2-3} \cline{5-6}
		& $b_{23}$      & $-0.234$    & & $b_{33}$      & $0.496$   \\ % \cline{2-3} \cline{5-6}
		& $b_{24}$      & $0.094$  & & $b_{34}$      & $0.331$  \\ % \cline{2-3} \cline{5-6}
		& $b_{25}$      & $-0.0254$   & & $b_{35}$      & $-0.641$  \\ \hline
		%
		\multirow{6}{*}{$p_{24}$} & $b_{40}$      & $0.379$  & \multirow{6}{*}{$p_{25}$} & $b_{50}$      & $-0.224$   \\ %\cline{2-3} \cline{5-6}
		& $b_{41}$      & $0.908$     &  & $b_{51}$      & $0.109$\\ %\cline{2-3} \cline{5-6}
		& $b_{42}$      & $-1.330$     &  & $b_{52}$      & $0.805$\\ % \cline{2-3} \cline{5-6}
		& $b_{43}$      & $-1.203$    &  & $b_{53}$      & $0.669$\\ %\cline{2-3} \cline{5-6}
		& $b_{44}$      & $0.215$     &  & $b_{54}$      & $-0.527$\\ % \cline{2-3} \cline{5-6}
		& $b_{45}$      & $0.683$   &  & $b_{55}$      & $-0.112$\\
		\hline
	\end{tabular}
\end{table}
%
\begin{table}[tb]
	\centering
	\caption{Coefficients for $\beta$.}
	\label{T:beta}
	\begin{tabular}{@{}cll}
		\hline
		%\multicolumn{6}{c}{$\kappa_p$ coefficients}           \\ \hline
		$p_{ij}$                  & \multicolumn{2}{c}{$b_{ik}$}  \\
		\hline
		\multirow{6}{*}{$p_{30}$} & $b_{00}$  & $0.538$
		
		\\ % \cline{2-3} \cline{5-6}
		& $b_{01}$      & $0.023$     \\ % \cline{2-3} \cline{5-6}
		& $b_{02}$      & $0.179$   	\\ % \cline{2-3} \cline{5-6}
		& $b_{03}$      & $-0.114$    \\ % \cline{2-3} \cline{5-6}
		& $b_{04}$      & $0.047$   	  \\ % \cline{2-3} \cline{5-6}
		& $b_{05}$      & $-0.034$   \\ \hline
		%
		\multirow{6}{*}{$p_{31}$} & $b_{10}$  & -0.152
		\\ % \cline{2-3} \cline{5-6}
		& $b_{11}$      & $0.065$    \\ % \cline{2-3} \cline{5-6}
		& $b_{12}$      & $0.277$      \\ % \cline{2-3} \cline{5-6}
		& $b_{13}$      & $0.017$      \\ % \cline{2-3} \cline{5-6}
		& $b_{14}$      & $-0.052$   \\ % \cline{2-3} \cline{5-6}
		& $b_{15}$      & $-0.082$     \\ \hline
		%
		
		\multirow{6}{*}{$p_{32}$} & $b_{20}$  & 0.585
		\\ % \cline{2-3} \cline{5-6}
		& $b_{21}$      & $-0.082$    \\ % \cline{2-3} \cline{5-6}
		& $b_{22}$      & $-0.280$      \\ % \cline{2-3} \cline{5-6}
		& $b_{23}$      & $0.116$      \\ % \cline{2-3} \cline{5-6}
		& $b_{24}$      & $0.011$   \\ % \cline{2-3} \cline{5-6}
		& $b_{25}$      & $0.044$     \\ \hline
		%	
	\end{tabular}
\end{table}
%
\subsection{Comparison of regression against Pareto data}
%
To compare the results from the tuning rule, some simulations were done to compare the original data against the results. The plant model is: 

\begin{equation}
P_1(s) = \frac{e^{-1.5\hat{s}}}{(\hat{s}+1)(0.5\hat{s}+1)}
\label{E:P1}
\end{equation}

Where $K=1$, $T=1$~s, $L=1.5$~s and $a = 0.5$. Table \ref{T:comparison} compares the results of the optimization against the results of using the proposed tuning rule. Arbitrarily, the values for $\delta$ and $\gamma$ were chosen as $\delta=1$ and $\gamma=1$. 
%
\begin{table}[tb]
	\centering
	\caption{Result comparative of the Pareto data against the fitted data, with $\delta = 1$ and $\gamma = 1$.}
	\label{T:comparison}
	\begin{tabular}{@{}K{0.25\columnwidth} K{0.25\columnwidth} K{0.25\columnwidth}@{}}
		\toprule
		$\bm{\theta}$ and cost functions & From Pareto & From Tuning rule\\
		\midrule
		$\kappa_p$	& $0.810$	& $0.793$ \\
		$\tau_i$	& $2.176$~s	& $2.113$~s	\\
		$\tau_d$	& $0.644$~s	& $0.720$~s \\
		$\beta$		& $1.000$	& $1.000$ \\
		$J_r$		& $2.689$ 	& $2.691$ \\
		$J_{di}$ 	& $2.687$ 	& $2.673$ \\
		$J_{do}$	& $2.689$	& $2.691$ \\
		$M_s$		& $1.9174$	& $1.9449$\\
		\bottomrule
	\end{tabular}
\end{table} 

From Table~\ref{T:comparison}, can be seen that the results obtained from the  tuning rule are close to those obtained directly from the Pareto. Plots for each method were drawn as shown in Fig.~\ref{F:firstsim}.
%%%ESTA FIGURA HAY QUE CENTRARLA MEJOR
%
\begin{figure}
	\centering
	\includegraphics[width=0.8\columnwidth]{servo2.png}
	\caption{Servo response for the Pareto results and tuning results.}
	\label{F:firstsim}
\end{figure}
%
The step response of the control signal is presented in Fig.~\ref{F:u1}.
%
\begin{figure}
	\centering
	\includegraphics[width=0.8\columnwidth]{u2.png}
	\caption{Control action response for ENCC results and regressions, for a reference step change.}
	\label{F:u1}
\end{figure}

---- Voy por acá ---------------
The comparison for an input-disturbance is presented in Fig.~\ref{F:di1}. The error between the tuning rule and the Pareto results has an \gls{iae} of $0.08$, showing that the polynomial equations are able to encapsulate the optimal tuning of the parameters. In Fig.~\ref{F:do1}, the response to the output disturbance is presented and, as it can be seen, the match is good as well, with an \gls{iae} value of $0.12$.
%
\begin{figure}
	\centering
	\includegraphics[width=0.8\columnwidth]{di2.png}
	\caption{Step input disturbance response for ENNC results and regressions results.}
	\label{F:di1}
\end{figure}
%
\begin{figure}
	\centering
	\includegraphics[width=0.8\columnwidth]{do2.png}
	\caption{Step output disturbance response for ENNC results and regressions results.}
	\label{F:do1}
\end{figure}
%

From this example, it is clear that the obtained polynomial equations effectively reproduce the behavior of the parameters found from the ENNC optimization.

The method is also tested for the extreme case scenarios for $\tau_0$ ($\tau_0=1$ and $\tau_0=2$), as given by the models:
%
\begin{equation}
P_F(\hat{s}) = \frac{e^{-\hat{s}}}{(\hat{s}+1)(0.5\hat{s}+1)},
\label{E:p2}
\end{equation}
%
\begin{equation}
P_S(\hat{s}) = \frac{e^{-2\hat{s}}}{(\hat{s}+1)(0.5\hat{s}+1)},
\label{E:p3}
\end{equation}
%
where $P_F(\hat{s})$ and $P_S(\hat{s})$ stand for the fastest and slowest time-delayed dominant test-bench plants, respectively.

For each of these plants, it was established that $\delta = 0.5$ and $\gamma = 0.5$, which is an intermediate case for degradation for both $\delta$ and $\gamma$ in the Pareto front. Results for $J_r$ for $P_F$ and $P_S$ are shown in Table \ref{T:T2}.

\begin{table}
	\centering
	\caption{Results for $J_{di}$, $J_{do}$ and $J_{r}$, using $\delta = 0.5$ and $\gamma = 0.5$.}
	\label{T:T2}
	\begin{tabular}{@{}K{0.25\columnwidth} K{0.25\columnwidth} K{0.25\columnwidth}@{}}
		\toprule
		$\bm{\theta}$ and IAE & For $P_F(s)$ & For $P_S(s)$\\
		\midrule
		$\kappa_p$	& $1.150$ 	& $0.742$ \\
		$\tau_i$ 	& $1.987$ s & $2.345$ s \\
		$\tau_d$ 	& $0.425$ s & $0.629$ s \\
		$\beta$ 	& $0.887$ 	& $0.919$ \\
		$J_r$ 		& $1.955$ 	& $3.360$ \\
		$J_{di}$ 	& $1.729$ 	& $3.162$ \\
		$J_{do}$ 	& $1.874$ 	& $3.237$ \\
		$M_s$		& $2.024$	& $1.976$ \\
		\bottomrule
	\end{tabular}
\end{table} 
%

From these examples, it is clear that the proposed methodology is able to produce Pareto-optimal controllers, for a large set of plants. The obtained dynamical response can be easily changed by the control engineer and, if the selection of $\delta$ and $\gamma$ is appropriate, the controller parameters are likely to produce a closed-loop system with a maximum sensitivity such that $M_s \leq 2$. The principal characteristic of the proposed methodology, is its ability to let the user select the dynamical behavior, taking into account multiple sources of disturbances, unlike other \gls{pid} tuning rules.

\subsection{Comparative of proposed tuning rule against uSORT2 tuning rule}
%
Servo and disturbance responses were obtained for the test-bench plant in \eqref{E:P1}, by using the proposed tuning rule, and then, by using the uSORT2 tuning method for \gls{2dof} \gls{pid} controllers. The objective is to compare both rules to see which one is more flexible in terms of robustness and performance. The uSORT2 method was selected for comparison given that it minimizes $J_{di}$ and find a sub-optimal solution for $J_r$ while keeping $M_s=2$.
%
%Three cases were considered, one in which $\delta$ and $\gamma$ were maximum as in the comparison made in Table~\ref{T:T4}, and the others were when $\delta$ and $\gamma$ have their minimum achievable value for the obtained Pareto front as in Table~\ref{T:T5} and Table~\ref{T:T6}.

Two cases were considered, one in which $\delta$ and $\gamma$ were maximum as in the comparison made in Table~\ref{T:T4}, and the others were when $\delta$ and $\gamma$ have their minimum achievable value for the obtained Pareto front as in Table~\ref{T:T5}.
%%
\begin{table}
	\centering
	\caption{Result comparative of the proposed tuning rule vs. uSORT2 tuning rule, using $\delta = 1$ and $\gamma = 1$.}
	\label{T:T4}
	\begin{tabular}{@{}K{0.25\columnwidth} K{0.25\columnwidth} K{0.25\columnwidth}@{}}
		\midrule
		$\bm{\theta}$ and IAEs & Proposed tuning rule & uSORT2 tuning rule \\
		\midrule
		$\kappa_p$ & 0.793 & 0.814 \\
		$\tau_i$ & 2.113 s & 1.676 s \\
		$\tau_d$ & 0.720 s & 0.775 s \\
		$\beta$ & 1.000 & 0.788 \\
		$J_r$ & 2.691 & 2.692 \\
		$J_{di}$ & 2.673 & 2.290 \\
		$J_{do}$ & 2.691 & 2.451 \\
		$M_{s}$ & 1.945 & 2.007 \\
		\bottomrule
	\end{tabular}
\end{table} 
%
\begin{table}
	\centering
	\caption{Result comparative of the proposed tuning rule vs. uSORT2 tuning rule, using $\delta = 0.302$ and $\gamma = 0$.}
	\label{T:T5}
	\begin{tabular}{@{}K{0.25\columnwidth} K{0.25\columnwidth} K{0.25\columnwidth}@{}}
		\midrule
		$\bm{\theta}$ parameters and IAEs & Proposed tuning rule & uSORT2 tuning rule \\
		\midrule
		$\kappa_p$ & 0.820 & 0.814 \\
		$\tau_i$ & 1.808 s & 1.676 s \\
		$\tau_d$ &  0.670 s & 0.775 s \\
		$\beta$ & 0.8261 & 0.788 \\
		$J_r$ & 2.653 & 2.692 \\
		$J_{di}$ & 2.307 & 2.290 \\
		$J_{do}$ & 2.431 & 2.451 \\
		$M_{s}$ & 1.935 & 2.007 \\
		\bottomrule
	\end{tabular}
\end{table}
%%
%\begin{table}
%\centering
%\caption{Result comparative of the proposed tuning rule vs. uSORT2 tuning rule, using $\delta = 0$ and $\gamma = 0.208$.}
%\label{T:T6}
%\begin{tabular}{@{}K{0.25\columnwidth} K{0.25\columnwidth} K{0.25\columnwidth}@{}}
%\midrule
%$\bm{\theta}$ parameters and IAEs & Proposed tuning rule & uSORT2 tuning rule \\
%\midrule
%$\kappa_p$ & 0.855 & 0.814 \\
%$\tau_i$ & 1.762 s & 1.676 s \\
%$\tau_d$ &  0.604 s & 0.775 s \\
%$\beta$ & 0.763 & 0.788 \\
%$J_r$ & 2.573 & 2.692 \\
%$J_{di}$ & 2.143 & 2.290 \\
%$J_{do}$ & 2.484 & 2.451 \\
%$M_{s}$ & 1.980 & 2.007 \\
%\bottomrule
%\end{tabular}
%\end{table}
%%
%%
\begin{figure}
	\centering
	\includegraphics[width=0.8\columnwidth]{./uSORT2.png}
	\caption{Servo input response for both proposed rule and uSORT2 rule.}
	\label{F:uSort}
\end{figure}
%%
%From the results obtained in Tables \ref{T:T4}, \ref{T:T5} and \ref{T:T6},
It is clear that for the two different variations of $\delta$ and $\gamma$, the minimum \gls{iae} for servo response was obtained according to the Pareto front. Also it is important to note that, for the proposed tuning rule, although variations of $\delta$ and $\gamma$ were performed, the maximum sensitivity remained under the value of $M_s \leq 2.0$. This means that given any two inputs within the established Pareto front, the proposed tuning rule gives the optimized value for $J_r$ and it manages to maintain the controller robustness. On the other hand, the uSORT2 tuning rule only focuses on the latter. %For the three given cases in tables \ref{T:T4}, \ref{T:T5} and \ref{T:T6}, the proposed tuning rule proves to be versatile not only in terms of robustness, but also in terms of performance choice.
The uSORT2 methodology provides excellent results, but it is focused on optimizing $J_{di}$. The proposed methodology, although more complex in its equations, gives the control engineers, the freedom to mold the dynamic response to their needs. As an example, the servo response of the controlled system with the proposed methodology and the uSORT is presented in Fig.~\ref{F:uSort}
%
\section{Database approach for the final tuning}
\label{sec:DatabaseMOOP}
To do
%